% Metódy inžinierskej práce

\documentclass[10pt,twoside,slovak,a4paper]{article}

\usepackage[slovak]{babel}
%\usepackage[T1]{fontenc}
\usepackage[IL2]{fontenc} % lepšia sadzba písmena Ľ než v T1
\usepackage[utf8]{inputenc}
\usepackage{graphicx}
\usepackage{url} % príkaz \url na formátovanie URL
\usepackage{hyperref} % odkazy v texte budú aktívne (pri niektorých triedach dokumentov spôsobuje posun textu)
\usepackage{cite}
\usepackage{graphicx}
%\usepackage{times}

\pagestyle{headings}

\title{Ako algoritmy ovplyvňujú naše voľby filmov, seriálov a hudby\thanks{Semestrálny projekt v predmete Metódy inžinierskej práce, ak. rok 20204/25, vedenie: Ing. Ivan Kaoustík}} % meno a priezvisko vyučujúceho na cvičeniach

\author{Marcel Novomesky\\[2pt]
	{\small Slovenská technická univerzita v Bratislave}\\
	{\small Fakulta informatiky a informačných technológií}\\
	{\small \texttt{xnovomesky@stuba.sk}}
	}

\date{\small 30. september 2024} % upravte



\begin{document}
\maketitle
\begin{abstract}


Primárnym zdrojom pre výber tejto témy bola štúdia o metódach odporúčania hudby. Tieto algoritmy analyzujú pužívatelské preferencie, a na základe histórie počúvania, hodnotenia a času stráveného pri jednotlivých žánroch vyberá najvhodnejšie odporúčania ktoré by sa mohli používateľovy páčiť. Tento článok \cite{9927924} ma inšpiroval širšie sa venovať týmto algoritmom a jejich podrobnému fungovaniu v praxi a uplatneniu. Hlavnú úlohu alebo teda problém ktorý rišia, je výber a odporúčanie hudby ktorá je vhodná pre daného používateľa. Práve preto sú aj tieto algoritmy klúčové pre fungovanie služieb ako napríklad Spotify alebo Youtube Music. Práve tieto firmy vyuižívajú podobné systémy na zaujatie používateľov a udržanie ich na platforme. Čím lepší odporúčiavací algoritmus, tým dlhšie si daná platforma dokáže udržať aktívných používateľov. Preto som sa aj venoval hybridnému odporúčaniu hudby. Tento systém sa od normalných systémov líši práve tým, že nevytrvára odporúčania iba na základe preferencie hudobného žánru používateľa, ale využíva aj takzvané kolaboratívne filtrovanie. Kolaboratívne filtrovanie je odporúčania na základe analýzy preferencií viacerých používateľov s podobnými zaujamamy o hudbu a žánre. Svoju pozornosť som venoval aj etickým aspektom týchto systémov, ako je aj ochrana súkromia či spracovanie používateľských dát. Dôležitá je aj transparentnosť o tom ako sú algoritmy používané a dáta spracované. 
\end{abstract}



\section{Úvod}
V dnešnej dobe zábavný priemysel čoraz viac využíva algoritmy na personalizované odporúčania filmov, seriálov a hudby. Preto som sa v mojej práci zameral na skúmanie toho, ako tieto systémy a algoritmy fungujú, aké dáta o používateľoch zbierajú a aké metódy sa využívajú na analýzu týchto dát. Zároveň sa pokúsim odpovedať na otázky, ako tieto alogritmy fungujú, či takéto odporúčania naozaj fungujú. 



\section{Potrebnosť týchto algoritmov} \label{nejaka}


\begin{figure*}[tbh]
\centering
%\includegraphics[scale=1.0]{diagram.pdf}
Aj text môže byť prezentovaný ako obrázok. Stane sa z neho označný plávajúci objekt. Po vytvorení diagramu zrušte znak \texttt{\%} pred príkazom \verb|\includegraphics| označte tento riadok ako komentár (tiež pomocou znaku \texttt{\%}).
\caption{Rozhodujúci argument.}
\label{f:rozhod}
\end{figure*}



\section{Hybridné algoritmy a ich fungovanie} \label{ina}

Základným problémom je teda\ldots{} Najprv sa pozrieme na nejaké vysvetlenie (časť~\ref{ina:nejake}), a potom na ešte nejaké (časť~\ref{ina:nejake}).\footnote{Niekedy môžete potrebovať aj poznámku pod čiarou.}

Môže sa zdať, že problém vlastne nejestvuje\cite{Coplien:MPD}, ale bolo dokázané, že to tak nie je~\cite{Czarnecki:Staged, Czarnecki:Progress}. Napriek tomu, aj dnes na webe narazíme na všelijaké pochybné názory\cite{PLP-Framework}. Dôležité veci možno \emph{zdôrazniť kurzívou}.


\subsection{Nejaké vysvetlenie} \label{ina:nejake}

Niekedy treba uviesť zoznam:

\begin{itemize}
\item jedna vec
\item druhá vec
	\begin{itemize}
	\item x
	\item y
	\end{itemize}
\end{itemize}

Ten istý zoznam, len číslovaný:

\begin{enumerate}
\item jedna vec
\item druhá vec
	\begin{enumerate}
	\item x
	\item y
	\end{enumerate}
\end{enumerate}


\subsection{Ešte nejaké vysvetlenie} \label{ina:este}

\paragraph{Veľmi dôležitá poznámka.}
Niekedy je potrebné nadpisom označiť odsek. Text pokračuje hneď za nadpisom.



\section{Spracovanie používatelských dát} \label{dolezita}




\section{Etické otázky} \label{dolezitejsia}

\begin{figure}[htbp]
  \centering
  \begin{minipage}{0.45\textwidth}
    \centering
    \includegraphics[width=\textwidth]{dig1.png}
    \caption{Popis prvého obrázku}
    \label{fig:dig1}
  \end{minipage}\hfill
  \begin{minipage}{0.45\textwidth}
    \centering
    \includegraphics[width=\textwidth]{dig2.png}
    \caption{Popis druhého obrázku}
    \label{fig:dig2}
  \end{minipage}
\end{figure}


\section{Záver} \label{zaver} % prípadne iný variant názvu



%\acknowledgement{Ak niekomu chcete poďakovať\ldots}


% týmto sa generuje zoznam literatúry z obsahu súboru literatura.bib podľa toho, na čo sa v článku odkazujete
\bibliography{literatura}
\bibliographystyle{plain} % prípadne alpha, abbrv alebo hociktorý iný
\end{document}
